% Very simple template for lab reports. Most common packages are already included.
\documentclass[a4paper]{article}
\pdfpagewidth=\paperwidth
\pdfpageheight=\paperheight
\usepackage[utf8]{inputenc} % Change according your file encoding
\usepackage{graphicx}
\usepackage{url}
\usepackage{listings}
\usepackage{xcolor}

\definecolor{mygray}{rgb}{0.5,0.5,0.5}

\lstset{
    basicstyle=\footnotesize,
    language=erlang,
    frame=single,
    numbers=left,
    numberstyle=\tiny\color{mygray},
    breaklines=true,
}

%opening
\title{Rudy: a small web server}
\author{Antonios Kouzoupis $<$antkou$@$kth.se$>$}
\date{September 17, 2014}

\begin{document}

\maketitle

\section{Introduction}

In this seminar our task is to write a very simple web server in Erlang. It will
be able to receive only \textit{GET} requests from a client and respond with a
message. I have implemented a file delivery functionality so it is closer to a
real web server. Some metrics will also be provided to evaluate the overall
performance of our small server.

After we finish this seminar we will be able to:
\begin{itemize}
    \item understand and use the socket API of Erlang
    \item understand the structure of a server process
    \item understand and describe the main aspects of HTTP protocol
    \item do some programming in Erlang
\end{itemize}

A web server, as a client--server architecture, is a distributed system in
terms that there are resources several users can share, resources that we
probably do not know where they reside physically. The communication
between the client and the server is message oriented using \textit{HTTP} as an
application layer protocol and \textit{TCP} as a transport layer protocol.

\section{Main problems and solutions}
The first task was to complete the TCP structure of the server. The source code
at Figure \ref{fig:tcp_server} illustrates the sequence of a connection. In line
3 we set up a socket to listen to port \emph{Port} with the options specified in
line 2. Then in case of an incoming connection, we accept the connection (lines
12 -- 17) and the \textit{request/1} function receives the request. In line 23
we parse the client's request, we prepare a response by calling the
\textit{reply/1} function and we send the reply in line 25. Finally we close the
connection to the client (line 29).\\

\begin{figure}
    \lstinputlisting[language=erlang]{tcp_con.erl}
    \caption{TCP Server}
    \label{fig:tcp_server}
\end{figure}

\begin{figure}
    \lstinputlisting[language=erlang]{reply.erl}
    \caption{reply/1 function}
    \label{fig:reply}
\end{figure}

The initial requirement was to build a web server which responds with a
\textbf{200 OK} status code to a client's request. I have extended that
requirement so that it can respond with a file requested by the client.

The first issue was to extract the file path from the \textit{URI} which was
trivial. Using pattern match I extracted the path and name of file without the
initial slash, code 47 in ASCII table and encapsulate it in the tuple that the
http parser returns.

Figure \ref{fig:reply} illustrates the \emph{reply/1} function which reads a
file and sends its content to the client. In line 1 we pattern patch to get
file name and in line 2 we use the built-in function of Erlang to read the
contents of file. \emph{read\_file/1} function returns a binary data object of
the contents so in line 5 we convert it to a list of characters and finally in
line 6 we call the \emph{ok/1} function which assemble the response.

The problem with delivering files was that most of the browsers after making
the initial request of the web page, they automatically made a second request to
get the \emph{favicon.ico}. Using Wireshark to intercept the packets, it was
clear that after each request for a web page, a second one was made for the
favicon and as a consequence Rudy did not crash but it produced a lot of error
messages. On the contrary, if we use a command line interface browser like
Elinks, there is no such problem.

\section{Evaluation}

In the following section I will provide some metrics taken regarding the average
response time of our web server for different scenarios. The metrics presented
in Table \ref{tab:metrics} were taken by the benchmark provided and making 100
request each time.

\begin{table}
    \centering
    \begin{tabular}{| l | c | c |}
        \hline
        \textbf{Scenario}       & \textbf{Time (ms)} & \textbf{Requests/sec}\\
        \hline
        \hline
        localhost/File exists   & 43.54     & 2296\\
        \hline
        localhost/File does not exist  & 60.55 & 1651\\
        \hline
        localhost/One request  & 0.749 & \\
        \hline
        Remote Host (Greece)    & 17002 &   6\\
        \hline
        Remote Host (wlan)      & 609   &  164\\
        \hline
        2 Remote Hosts (wlan)   & 1293  &  77\\ 
        \hline
    \end{tabular}
    \caption{Metrics}
    \label{tab:metrics}
\end{table}

As we expected, the response time when running the benchmark from localhost is
much lower than running the benchmark from another machine in the same network.
Very interesting is the fact that the response time measured in case of
not finding the requested file is greater than the case when the file is found. If Rudy does not find a file
it prints out an error message in terminal, so probably all these I/O interrupts
produce some delay. Also, I run the Rudy server in a machine in Greece,
introducing a significant amount of latency which leads us to high response time.
Finally, two of my colleagues run the benchmark for 100 requests simultaneously
which obviously resulted in a delayed response because of the single threaded
architecture of our web server. 

\section{Conclusions}

To conclude, in this seminar we have learnt the basics of HTTP protocol, how the
request and reply is constructed and how to write a simple TCP server. We also
had the opportunity to dive into Erlang and functional programming aspects.

\end{document}
